%
% Layout retirado de http://www.di.uminho.pt/~prh/curplc09.html#notas
%
\documentclass{report}
    %encoding
    %--------------------------------------
    \usepackage[utf8]{inputenc}
    \usepackage[T1]{fontenc}
    %--------------------------------------
    
    %Portuguese-specific commands
    %--------------------------------------
    \usepackage[portuguese]{babel}
    %--------------------------------------
    
    %Hyphenation rules
    %--------------------------------------
    \usepackage{hyphenat}
    \hyphenation{mate-mática recu-perar}
    %--------------------------------------
    
    \usepackage{url}
    \usepackage{enumerate}
    \usepackage{graphicx}
    
    %\usepackage{alltt}
    %\usepackage{fancyvrb}
    \usepackage{listings}
    %LISTING - GENERAL
    \lstset{
        basicstyle=\small,
        numbers=left,
        numberstyle=\tiny,
        numbersep=5pt,
        breaklines=true,
        frame=tB,
        mathescape=true,
        escapeinside={(*@}{@*)}
    }
    %
    %\lstset{ %
    %   language=Java,                          % choose the language of the code
    %   basicstyle=\ttfamily\footnotesize,      % the size of the fonts that are used for the code
    %   keywordstyle=\bfseries,                 % set the keyword style
    %   %numbers=left,                          % where to put the line-numbers
    %   numberstyle=\scriptsize,                % the size of the fonts that are used for the line-numbers
    %   stepnumber=2,                           % the step between two line-numbers. If it's 1 each line
    %                                           % will be numbered
    %   numbersep=5pt,                          % how far the line-numbers are from the code
    %   backgroundcolor=\color{white},          % choose the background color. You must add \usepackage{color}
    %   showspaces=false,                       % show spaces adding particular underscores
    %   showstringspaces=false,                 % underline spaces within strings
    %   showtabs=false,                         % show tabs within strings adding particular underscores
    %   frame=none,                             % adds a frame around the code
    %   %abovecaptionskip=-.8em,
    %   %belowcaptionskip=.7em,
    %   tabsize=2,                              % sets default tabsize to 2 spaces
    %   captionpos=b,                           % sets the caption-position to bottom
    %   breaklines=true,                        % sets automatic line breaking
    %   breakatwhitespace=false,                % sets if automatic breaks should only happen at whitespace
    %   title=\lstname,                         % show the filename of files included with \lstinputlisting;
    %                                           % also try caption instead of title
    %   escapeinside={\%*}{*)},                 % if you want to add a comment within your code
    %   morekeywords={*,...}                    % if you want to add more keywords to the set
    %}
    
    \usepackage{xspace}
    
    \parindent=0pt
    \parskip=2pt
    
    \setlength{\oddsidemargin}{-1cm}
    \setlength{\textwidth}{18cm}
    \setlength{\headsep}{-1cm}
    \setlength{\textheight}{23cm}
    
    \def\darius{\textsf{Darius}\xspace}
    \def\antlr{\texttt{AnTLR}\xspace}
    \def\pl{\emph{Processamento de Linguagens}\xspace}
    
    \def\titulo#1{\section{#1}}
    \def\super#1{{\em Supervisor: #1}\\ }
    \def\area#1{{\em \'{A}rea: #1}\\[0.2cm]}
    \def\resumo{\underline{Resumo}:\\ }
    
    
    %%%%\input{LPgeneralDefintions}
    
    \title{Processamento de Linguagens (3º ano do MiEI)\\ \textbf{Trabalho Prático 1}\\ Flex}
    \author{Joana Cruz\\ (A76270) \and Rui Azevedo\\ (A80789) \and Maurício Salgado\\ (A71407) }
    \date{\today}
    
    \begin{document}
    
    \maketitle
    
    \begin{abstract}
    
    O presente trabalho tem como objetivo aumentar a experiência no uso do ambiente Linux, aumentar capacidade de escrever Expressões Regulares (ER) e, a partir destas,  desenvolver Processadores de Linguagens Regulares.
    A ferramenta de processamento de texto utilizada é o C Flex que, por sua vez, também utiliza o GCC. Todos os objetivos inicialmente propostos foram cumpridos.
    
    \end{abstract}
    
    \tableofcontents
    
    \chapter{Introdu\c{c}\~ao} \label{intro}
    
    \section*{Introdu\c{c}\~ao} \
    
    A diversidade e quantidade de informação nos dias de hoje é cada vez maior e mais dispersa, está em todo o lado em grandes quantidades, a todo o momento. Posto isto, torna-se ainda mais necessária uma linguagem de programação como o C Flex, que permite filtrar, de maneira facilitada, a informação essencial num ficheiro em que os dados estejam dispersos e ainda tratar essa informação. O trabalho apresentado na unidade curricular de Processamento de Linguagens tem como objetivo usar esta ferramenta para filtrar um conjunto de dados fornecidos, extraindo desses dados informação relevante.
    
    \section*{Sele\c{c}\~ao de enunciados} \
    
    
    \section*{Objetivos}
    Para o desenvolvimento deste trabalho foram definidos os seguintes objetivos:

    \newpage
    \section*{Estrutura do Relatório} \
    

    
    \section*{Características dos Dados, Padr\~oes de Frase e Decisões}
    
    
    \chapter{Implementação e Ações Semânticas} \label{ae}

    
    \section{Apresentação do Output}
    Para tratamento dos dados de input e respectiva apresentação da informação tratada foi desenvolvida a seguinte main:
    

    
    \chapter{Apresentação de Resultados}

    \section{Input}
    \begin{verbatim}

    \end{verbatim}

    \newpage
    \section{Output}

    \chapter{Análise de Resultados}
   
    \chapter{Conclusão} \label{concl}
   
    \appendix
    \chapter{Código FLEX}

    \begin{verbatim}

    \end{verbatim}
    
    \bibliographystyle{alpha}
    \bibliography{relprojLayout}
    
    \end{document}